\documentclass[
]{jss}

\usepackage[utf8]{inputenc}

\providecommand{\tightlist}{%
  \setlength{\itemsep}{0pt}\setlength{\parskip}{0pt}}

\author{
FirstName LastName\\University/Company \And Second Author\\Affiliation
}
\title{A Capitalized Title: Something about a Package \pkg{foo}}

\Plainauthor{FirstName LastName, Second Author}
\Plaintitle{A Capitalized Title: Something about a Package foo}
\Shorttitle{\pkg{foo}: A Capitalized Title}

\Abstract{
The abstract of the article.
}

\Keywords{keywords, not capitalized, \proglang{Java}}
\Plainkeywords{keywords, not capitalized, Java}

%% publication information
%% \Volume{50}
%% \Issue{9}
%% \Month{June}
%% \Year{2012}
%% \Submitdate{}
%% \Acceptdate{2012-06-04}

\Address{
    FirstName LastName\\
  University/Company\\
  First line Second line\\
  E-mail: \email{name@company.com}\\
  URL: \url{http://rstudio.com}\\~\\
    }


% Pandoc header

\usepackage{amsmath}

\begin{document}

\hypertarget{introduction}{%
\section{Introduction}\label{introduction}}

This template demonstrates some of the basic latex you'll need to know
to create a JSS article.

\hypertarget{code-formatting}{%
\subsection{Code formatting}\label{code-formatting}}

Don't use markdown, instead use the more precise latex commands:

\begin{itemize}
\item
  \proglang{Java}
\item
  \pkg{plyr}
\item
  \code{print("abc")}
\end{itemize}

\hypertarget{r-code}{%
\section{R code}\label{r-code}}

Can be inserted in regular R markdown blocks.

\hypertarget{mvn}{%
\subsection{MVN}\label{mvn}}

\begin{CodeChunk}

\begin{CodeInput}
R> library(CuSMC)
R> mu = c(0, 0)
R> sigma = matrix(c(1, 0, 0, 1), nrow=2)
R> CuSMC::MVN(mu, sigma)
\end{CodeInput}

\begin{CodeOutput}
[1] -1.161507 -4.146704
\end{CodeOutput}
\end{CodeChunk}

\hypertarget{mvnpdf}{%
\subsection{MVNPDF}\label{mvnpdf}}

\begin{CodeChunk}

\begin{CodeInput}
R> library(CuSMC)
R> x = c(0, 0)
R> mu = c(0, 0)
R> sigma = matrix(c(1, 0, 0, 1), nrow=2)
R> CuSMC::MVNPDF(x, mu, sigma)
\end{CodeInput}

\begin{CodeOutput}
[1] 0.1591549
\end{CodeOutput}
\end{CodeChunk}

\hypertarget{mvt}{%
\subsection{MVT}\label{mvt}}

\begin{CodeChunk}

\begin{CodeInput}
R> library(CuSMC)
R> mu = c(0, 0, 0)
R> sigma = diag(3)
R> nu = 3.0
R> CuSMC::MVT(mu, sigma, nu)
\end{CodeInput}

\begin{CodeOutput}
[1]  3.091552  1.811468 -2.445105
\end{CodeOutput}
\end{CodeChunk}

\hypertarget{mvtpdf}{%
\subsection{MVTPDF}\label{mvtpdf}}

\begin{CodeChunk}

\begin{CodeInput}
R> library(CuSMC)
R> x = c(0, 0, 0)
R> mu = c(0, 0, 0)
R> sigma = diag(3)
R> nu = 3.0
R> CuSMC::MVTPDF(x, mu, sigma, nu)
\end{CodeInput}

\begin{CodeOutput}
[1] 0.07799708
\end{CodeOutput}
\end{CodeChunk}

\hypertarget{metropolis-hastings}{%
\subsection{Metropolis-Hastings}\label{metropolis-hastings}}

\begin{CodeChunk}

\begin{CodeInput}
R> library(CuSMC)
R> N = 100
R> w = rnorm(N, 0, 1)
R> B = 10
R> CuSMC::metropolis_hastings(w, N, B)
\end{CodeInput}

\begin{CodeOutput}
  [1] 84 11 39 24 17 29 48 93 72 80 67 29 85 39  4 87 16 74 12 87 16 48 55 29 51
 [26] 51 21 51 58 51 30 30 84 40 33 97 40 58 77 39 13 27 85 17 86 32 16  3 74 51
 [51]  9 58 24 80 28 57 54 54 33 79 67 96  5 22 42 86  0 83 79 40 70 40 49 28 18
 [76] 38 13 84 42 31 81 15 44  0 32 77 54 83 77 32 99 77 86 22 48 29  9  9 48 84
\end{CodeOutput}
\end{CodeChunk}

\hypertarget{installation}{%
\subsection{Installation}\label{installation}}

\begin{CodeChunk}

\begin{CodeInput}
R> install.packages("devtools")
R> install.packages("githubinstall")
\end{CodeInput}
\end{CodeChunk}

\hypertarget{installation-2}{%
\subsection{Installation 2}\label{installation-2}}

\begin{CodeChunk}

\begin{CodeInput}
R> library(devtools)
R> library(githubinstall)
R> githubinstall("CuSMC")
R> library(CuSMC)
\end{CodeInput}
\end{CodeChunk}

\hypertarget{installation-2-1}{%
\subsection{Installation 2}\label{installation-2-1}}

\begin{CodeChunk}

\begin{CodeInput}
R> library(devtools)
R> library(githubinstall)
R> githubinstall("CuSMC")
\end{CodeInput}
\end{CodeChunk}



\end{document}


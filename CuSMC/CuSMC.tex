\documentclass[
]{jss}

\usepackage[utf8]{inputenc}

\providecommand{\tightlist}{%
  \setlength{\itemsep}{0pt}\setlength{\parskip}{0pt}}

\author{
FirstName LastName\\University/Company \And Second Author\\Affiliation
}
\title{A Capitalized Title: Something about a Package \pkg{foo}}

\Plainauthor{FirstName LastName, Second Author}
\Plaintitle{A Capitalized Title: Something about a Package foo}
\Shorttitle{\pkg{foo}: A Capitalized Title}

\Abstract{
The abstract of the article.
}

\Keywords{keywords, not capitalized, \proglang{Java}}
\Plainkeywords{keywords, not capitalized, Java}

%% publication information
%% \Volume{50}
%% \Issue{9}
%% \Month{June}
%% \Year{2012}
%% \Submitdate{}
%% \Acceptdate{2012-06-04}

\Address{
    FirstName LastName\\
  University/Company\\
  First line Second line\\
  E-mail: \email{name@company.com}\\
  URL: \url{http://rstudio.com}\\~\\
    }


% Pandoc header

\usepackage{amsmath}

\begin{document}

\hypertarget{introduction}{%
\section{Introduction}\label{introduction}}

This template demonstrates some of the basic latex you'll need to know
to create a JSS article.

\hypertarget{code-formatting}{%
\subsection{Code formatting}\label{code-formatting}}

Don't use markdown, instead use the more precise latex commands:

\begin{itemize}
\item
  \proglang{Java}
\item
  \pkg{plyr}
\item
  \code{print("abc")}
\end{itemize}

\hypertarget{r-code}{%
\section{R code}\label{r-code}}

Can be inserted in regular R markdown blocks.

\hypertarget{mvn}{%
\subsection{MVN}\label{mvn}}

\begin{CodeChunk}

\begin{CodeInput}
R> library(CuSMC)
R> mu = c(0, 0)
R> sigma = matrix(c(1, 0, 0, 1), nrow=2)
R> CuSMC::MVN(mu, sigma)
\end{CodeInput}

\begin{CodeOutput}
[1] 1.931198 1.741011
\end{CodeOutput}
\end{CodeChunk}

\hypertarget{mvnpdf}{%
\subsection{MVNPDF}\label{mvnpdf}}

\begin{CodeChunk}

\begin{CodeInput}
R> library(CuSMC)
R> x = c(0, 0)
R> mu = c(0, 0)
R> sigma = matrix(c(1, 0, 0, 1), nrow=2)
R> CuSMC::MVNPDF(x, mu, sigma)
\end{CodeInput}

\begin{CodeOutput}
[1] 0.1591549
\end{CodeOutput}
\end{CodeChunk}

\hypertarget{mvt}{%
\subsection{MVT}\label{mvt}}

\begin{CodeChunk}

\begin{CodeInput}
R> library(CuSMC)
R> mu = c(0, 0, 0)
R> sigma = diag(3)
R> nu = 3.0
R> CuSMC::MVT(mu, sigma, nu)
\end{CodeInput}

\begin{CodeOutput}
[1] -0.3027183 -2.4256343  2.3695596
\end{CodeOutput}
\end{CodeChunk}

\hypertarget{mvtpdf}{%
\subsection{MVTPDF}\label{mvtpdf}}

\begin{CodeChunk}

\begin{CodeInput}
R> library(CuSMC)
R> x = c(0, 0, 0)
R> mu = c(0, 0, 0)
R> sigma = diag(3)
R> nu = 3.0
R> CuSMC::MVTPDF(x, mu, sigma, nu)
\end{CodeInput}

\begin{CodeOutput}
[1] 0.07799708
\end{CodeOutput}
\end{CodeChunk}

\hypertarget{metropolis-hastings}{%
\subsection{Metropolis-Hastings}\label{metropolis-hastings}}

\begin{CodeChunk}

\begin{CodeInput}
R> library(CuSMC)
R> N = 100
R> w = rnorm(N, 0, 1)
R> B = 10
R> CuSMC::metropolis_hastings(w, N, B)
\end{CodeInput}

\begin{CodeOutput}
  [1] 50 66 69 53 48 11 44 55 66 89 61 33 22  6 64 64 70 67 13 46 34 18 20 92 64
 [26] 89 64 55 94 81 18 20 11 23 84 94 74 11 95 89 91 91  2 39 77 33 69 97 34 47
 [51] 72 74 88 88 87 88 48 29  3 95 89 20 64 92 32 88 17 25 87  0 89 94 99 64  2
 [76] 33  5  9 18  3 57 64 44 71 36 30  2 25 55 55 47 43 44 92 73 60 47 69 41 47
\end{CodeOutput}
\end{CodeChunk}



\end{document}

